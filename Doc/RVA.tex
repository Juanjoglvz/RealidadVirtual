\documentclass{pre-tfg}

\usepackage{listings}
\usepackage{longtable}
\usepackage{formular}
\usepackage{mathtools}
\usepackage[pdftex]{graphicx}
\usepackage{rotating}
\usepackage[utf8]{inputenx}
\usepackage[T1]{fontenc} % Codificación de salida 
\usepackage{marvosym}
\showhelp  % comenta o borra para eliminar ayudas

\title{Programación e interacción. Lenguaje C\# en Unity 3D, control de elementos multimedia y 3D.}
\author{Javier Córdoba Romero y Juan José Corroto Martin}
\advisorFirst{Javier Alonso Albusac Jiménez}
%\advisorSecond{}
%\intensification{(INTENSIFICACIÓN)}
\docdate{2019}{Diciembre}


\definecolor{bluekeywords}{rgb}{0,0,1}
\definecolor{greencomments}{rgb}{0,0.5,0}
\definecolor{redstrings}{rgb}{0.64,0.08,0.08}
\definecolor{xmlcomments}{rgb}{0.5,0.5,0.5}
\definecolor{types}{rgb}{0.17,0.57,0.68}

\usepackage{listings}
\lstset{language=[Sharp]C,
captionpos=b,
%numbers=left, %Nummerierung
%numberstyle=\tiny, % kleine Zeilennummern
frame=lines, % Oberhalb und unterhalb des Listings ist eine Linie
showspaces=false,
showtabs=false,
breaklines=true,
showstringspaces=false,
breakatwhitespace=true,
escapeinside={(*@}{@*)},
commentstyle=\color{greencomments},
morekeywords={partial, var, value, get, set},
keywordstyle=\color{bluekeywords},
stringstyle=\color{redstrings},
basicstyle=\ttfamily\small,
}

\usepackage[
	backend=biber, 		% Backend
	sorting=none,
	defernumbers=true, 	% Para numerar al final
	style=numeric-comp, % Estilo numérico condensado
	% Descomentar las opciones siguientes para bibliografía multilingüe
%	autolang=other, 	% Requerido para opción multilingüe
%	language=auto   	% Requerido para opción multilingüe
]{biblatex}


% Línea añadida para eliminar el idioma de la fuente bibliográfica.
\AtEveryBibitem{\clearfield{note} \clearlist{language}}
% OJO: Editar si se cambia el fichero de bibliografía. 
\addbibresource{RVA.bib} 	% Fichero de bibliografía.
%\usepackage[autostyle]{csquotes}


\renewcommand*\lstlistingname{Listado}

\begin{document}

\maketitle
\tableofcontents

\newpage
\section{C\#}

\begin{lstlisting}[caption=Hello world]
public class Cosa
{
	public static void main(String[] args)
	{
		Console.WriteLine("Hello World");
	}
}
\end{lstlisting}

Citacion: \cite{Bow93}.

\subsection{Programación basada en eventos}

\section{Uso de C\# en Unity}

La primera versión del motor de juegos Unity fue lanzada en el 2005, desde entonces mucho ha cambiado en el motor, pasando por el motor de scripting, nuevas tecnologías gráficas y el modelo de licencias.

El lenguaje de scripting de Unity pasó por varias etapas, la primera de ellas fue \emph{Boo}, un lenguaje orientado a objetos usando el \emph{Common Language Runtime} (CLR) de .NET Framework, tecnología similar a la usada por \emph{Java} en su máquina virtual. La siguiente etapa pasó por usar una variante de \emph{Javascript} como lenguaje de scripting y, por último, se pasó a usar C\# como lenguaje de scripting con \emph{Visual Studio} como uno de los IDEs soportados.

C\# es un lenguaje de programación orientado a objetos y ejecutado sobre el \emph{Common Language Runtime}, usando principalmente con tipos estáticos aunque con soporte para tipos dinámicos, también se puede enfocar a una programación basada en eventos gracias a sus \emph{delegados}.

Ha conseguido obtener la condición de estándar ISO, su últimya revisión fue en 2018, con referencia: ISO 23270:2018\footnote{\url{https://www.iso.org/standard/75178.html}}

Actualmente Unity soporta más de 20 plataformas\footnote{\url{https://unity3d.com/es/unity/features/multiplatform}}, entre las que más destacan podemos encontrar: Windows, Linux, Mac, Playstation 4, Xbox One, Nintendo 3Ds, Oculus Rift, Android, iOS, WebGL, ARKit y ARCore.

C\# en Unity también permite interaccionar con los diferentes componentes del motor, como por ejemplo \emph{Mecanim}, su sistema de animación, su sistema de físicas o su sistema de componentes, lo que proporciona un control total sobre el motor y no sólo tareas de scripting in-game.

\subsection{Estructura de clases de Unity}

\subsection{Manejo de objetos de Unity desde C\#}

\section{Ejemplos}

\section{Conclusion}


\section{Referencias}

\clearpage

%\phantomsection  % OJO: Ojo necesario con hyperref.
\addcontentsline{toc}{chapter}{\refname} % Añade la bibliografía al Índice de contenidos.
%---
% Opción 1: Bibliografía con todas las fuentes en un apartado.
%---
\printbibliography

\end{document}

% Local Variables:
% coding: utf-8
% mode: flyspell
% ispell-local-dictionary: "castellano8"
% mode: latex
% End:
