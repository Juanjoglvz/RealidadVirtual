\documentclass{pre-tfg}

\usepackage{listings}
\usepackage{longtable}
\usepackage{formular}
\usepackage{mathtools}
\usepackage[pdftex]{graphicx}
\usepackage{rotating}
\usepackage[utf8]{inputenx}
\usepackage[T1]{fontenc} % Codificación de salida 
\usepackage{marvosym}
\showhelp  % comenta o borra para eliminar ayudas

\title{Programación e interacción. Lenguaje C\# en Unity 3D, control de elementos multimedia y 3D.}
\author{Javier Córdoba Romero y Juan José Corroto Martin}
\advisorFirst{Javier Alonso Albusac Jiménez}
%\advisorSecond{}
%\intensification{(INTENSIFICACIÓN)}
\docdate{2019}{Diciembre}


\DeclareGraphicsExtensions{.pdf,.png,.jpg}

\usepackage{color}
\definecolor{gray97}{gray}{.97}
\definecolor{gray75}{gray}{.75}
\definecolor{gray45}{gray}{.45}

\lstset{ frame=Ltb,
     framerule=0pt,
     aboveskip=0.5cm,
     framextopmargin=3pt,
     framexbottommargin=3pt,
     framexleftmargin=0.4cm,
     framesep=0pt,
     rulesep=.4pt,
     backgroundcolor=\color{gray97},
     rulesepcolor=\color{black},
     %
     stringstyle=\ttfamily,
     showstringspaces = false,
     basicstyle=\small\ttfamily,
     commentstyle=\color{gray45},
     keywordstyle=\bfseries,
     %
     numbers=left,
     numbersep=15pt,
     numberstyle=\tiny,
     numberfirstline = false,
     breaklines=true,
     keepspaces=true,
   }

% minimizar fragmentado de listados
\lstnewenvironment{listing}[1][]
   {\lstset{#1}\pagebreak[0]}{\pagebreak[0]}

\lstdefinestyle{consola}
   {basicstyle=\scriptsize\bf\ttfamily,
    backgroundcolor=\color{gray75},
   }

\lstdefinestyle{C}
   {language=C,
   }


\renewcommand*\lstlistingname{Listado}

\begin{document}

\maketitle
\tableofcontents

\newpage
\section{Unity y C\#}

La primera versión del motor de juegos Unity fue lanzada en el 2005, desde entonces mucho ha cambiado en el motor, pasando por el motor de scripting, nuevas tecnologías gráficas y el modelo de licencias.

El lenguaje de scripting de Unity pasó por varias etapas, la primera de ellas fue \emph{Boo}, un lenguaje orientado a objetos usando el \emph{Common Language Runtime} (CLR) de .NET Framework, tecnología similar a la usada por \emph{Java} en su máquina virtual. La siguiente etapa pasó por usar una variante de \emph{Javascript} como lenguaje de scripting y, por último, se pasó a usar C\# como lenguaje de scripting con \emph{Visual Studio} como uno de los IDEs soportados.

C\# es un lenguaje de programación orientado a objetos y ejecutado sobre el \emph{Common Language Runtime}, usando principalmente con tipos estáticos aunque con soporte para tipos dinámicos, también se puede enfocar a una programación basada en eventos gracias a sus \emph{delegados}.

Ha conseguido obtener la condición de estándar ISO, su últimya revisión fue en 2018, con referencia: ISO 23270:2018\footnote{\url{https://www.iso.org/standard/75178.html}}

Actualmente Unity soporta más de 20 plataformas\footnote{\url{https://unity3d.com/es/unity/features/multiplatform}}, entre las que más destacan podemos encontrar: Windows, Linux, Mac, Playstation 4, Xbox One, Nintendo 3Ds, Oculus Rift, Android, iOS, WebGL, ARKit y ARCore.

C\# en Unity también permite interaccionar con los diferentes componentes del motor, como por ejemplo \emph{Mecanim}, su sistema de animación, su sistema de físicas o su sistema de componentes, lo que proporciona un control total sobre el motor y no sólo tareas de scripting in-game.

\section{Uso de C\# en Unity}

\section{Manejo de objetos de Unity desde C\#}

\section{Ejemplos}



\end{document}

% Local Variables:
% coding: utf-8
% mode: flyspell
% ispell-local-dictionary: "castellano8"
% mode: latex
% End:
