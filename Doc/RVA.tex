\documentclass{pre-tfg}

\usepackage{listings}
\usepackage{longtable}
\usepackage{formular}
\usepackage{mathtools}
\usepackage[pdftex]{graphicx}
\usepackage{rotating}
\usepackage[utf8]{inputenx}
\usepackage[T1]{fontenc} % Codificación de salida 
\usepackage{marvosym}
\showhelp  % comenta o borra para eliminar ayudas

\title{Programación e interacción. Lenguaje C\# en Unity 3D, control de elementos multimedia y 3D.}
\author{Javier Córdoba Romero y Juan José Corroto Martin}
\advisorFirst{Javier Alonso Albusac Jiménez}
%\advisorSecond{}
%\intensification{(INTENSIFICACIÓN)}
\docdate{2019}{Diciembre}


\definecolor{bluekeywords}{rgb}{0,0,1}
\definecolor{greencomments}{rgb}{0,0.5,0}
\definecolor{redstrings}{rgb}{0.64,0.08,0.08}
\definecolor{xmlcomments}{rgb}{0.5,0.5,0.5}
\definecolor{types}{rgb}{0.17,0.57,0.68}

\usepackage{listings}
\lstset{language=[Sharp]C,
captionpos=b,
%numbers=left, %Nummerierung
%numberstyle=\tiny, % kleine Zeilennummern
frame=lines, % Oberhalb und unterhalb des Listings ist eine Linie
numbers=left,
showspaces=false,
showtabs=false,
breaklines=true,
showstringspaces=false,
breakatwhitespace=true,
escapeinside={(*@}{@*)},
commentstyle=\color{greencomments},
morekeywords={partial, var, value, get, set},
keywordstyle=\color{bluekeywords},
stringstyle=\color{redstrings},
basicstyle=\ttfamily\footnotesize
}
\renewcommand*\lstlistingname{Listado}

\usepackage[
	backend=biber, 		% Backend
	sorting=none,
	defernumbers=true, 	% Para numerar al final
	style=numeric-comp, % Estilo numérico condensado
	% Descomentar las opciones siguientes para bibliografía multilingüe
%	autolang=other, 	% Requerido para opción multilingüe
%	language=auto   	% Requerido para opción multilingüe
]{biblatex}


% Línea añadida para eliminar el idioma de la fuente bibliográfica.
\AtEveryBibitem{\clearfield{note} \clearlist{language}}
% OJO: Editar si se cambia el fichero de bibliografía. 
\addbibresource{RVA.bib} 	% Fichero de bibliografía.
%\usepackage[autostyle]{csquotes}

\begin{document}

\maketitle
\tableofcontents

\newpage
\section{C\#}

C\# nació en el 2000 por parte de Microsoft enmarcado dentro del .NET Framework, un framework que permite reutilizar código escrito para otros lenguajes que también compilen para .NET.

.NET Framework se fue expandiendo con los años hasta llegar al 2016 con el desarrollo de .NET Core, una alternativa de código abierto y usable tanto en Windows, Linux y macOS, haciendo que los programas escritos para este framework pudiesen ser ejecutados en la mayoría de sistemas operativos.

C\# es un lenguaje de programación orientado a objetos con tipado fuerte y con recolector de basura ya integrado. La sintaxis de C\# está influenciada por la de C, C++ y Java.

\subsection{Tipos de Variable}

En C\# existen 4 categorías de variables:

\begin{itemize}
	\item Valor
	\item Referencia
	\item Parámetros genéricos
	\item Punteros
\end{itemize}

\paragraph{Tipos de variables por valor}

Los \texttt{structs} y la mayoría de tipos incluidos por defecto pertenecen a esta categoría, exceptuando al tipo \texttt{string}, lo mismo ocurre en Java.

Un ejemplo de tipo pasado por valor está en el Listado \ref{lst:valuedtypes}, donde se ve que aunque el valor de \texttt{x} sea asignado a \texttt{y} y posteriormente el valor de \texttt{x} cambie, el valor de \texttt{y} no lo hace.

\begin{spacing}{1}
\begin{lstlisting}[float=htbp, caption=Ejemplo de tipo de variable por valor, label=lst:valuedtypes]
static void Main()
{
	int x = 4;	
	int y = x;            // Se copia el (*@\bfseries valor@*)
	
	Console.WriteLine(x); // 4
	Console.WriteLine(y); // 4
	
	y = 7;
		
	Console.WriteLine(x); // 4
	Console.WriteLine(y); // 7
}
\end{lstlisting}
\end{spacing}

\paragraph{Tipos de variables por referencia}

Mientras que los tipos de variables por valor están compuestos por sólo una parte, el valor, los tipos por referencia están compuestos por dos partes: un objeto y la referencia a ese objeto. El valor que se almacena en la variable es la referencia a ese objeto, el objeto se creará y almacenará independientemente de la referencia cuando se utilice el operator \texttt{new}.

Un ejemplo de tipo de variable por referencia es la clase \texttt{Point2D}, que almacena el valor de un punto en el espacio 2D, el Listado \ref{lst:referencetypes} ilustra un ejemplo de el comportamiento de un tipo de variable por referencia.

\begin{spacing}{1}
\begin{lstlisting}[float=htbp, caption=Ejemplo de tipo de variable por referencia, label=lst:referencetypes]
public class Point2D 
{ 
	public int x;
	public int y;
}

static void Main()
{
	Point2D p1 = new Point2D();
	p1.x = 4;
	Point2D p2 = p1;         // Se copia la (*@\bfseries referencia@*)
	
	Console.WriteLine(p1.x); // 4
	Console.WriteLine(p2.y); // 4
	
	p1.x = 7;
		
	Console.WriteLine(x);    // 7
	Console.WriteLine(y);    -// 7
}
\end{lstlisting}
\end{spacing}

\subsubsection{Arrays}

En C\# los arrays están completamente soportados y gestionados por el lenguaje, sin que el programador tenga que preocuparse por la gestión de la memoria o de cómo se guardan en memoria.

Los arrays siempre se guardan en un bloque contiguo de memoria haciendo que los accesos a sucesivos elementos de un array sean muy eficientes.

Los elementos de un array siempre se inicializarán con un valor por defecto, este valor depende del tipo de variable del array.

Si el tipo de variable del array es por valor, el valor por defecto es cero en el caso de los tipos numéricos, \texttt{false} para el tipo booleano, en el caso de los structs, se inicializan con el valor predeterminado de cada elemento, como se puede ver en el Listado \ref{lst:arraytyped}.

\begin{spacing}{1}
\begin{lstlisting}[float=htbp, caption=Ejemplo de array con tipo de variable por valor, label=lst:arraytyped]
static void Main()
{
	int[] x = new int[5];    // Se crea un array de 5 enteros
	x[1] = 2;				 // Se asigna el valor 2 al segundo elemento del array
		
	Console.WriteLine(x[0]);    // 0
	Console.WriteLine(x[1]);    // 2
	Console.WriteLine(x[2]);    // 0
	Console.WriteLine(x[3]);    // 0
	Console.WriteLine(x[4]);    // 0
}
\end{lstlisting}
\end{spacing}

Sin embargo, si el tipo de variable del array es por referencia, el valor por defecto es \texttt{null} ya que , como se ha comentado anteriormente, lo que se almacena en la variable es la referencia al objeto.

La consecuencia de esto es que un array de tipo de variable por referencia necesita ser inicializado después de declarar el array, como se puede observar en el Listado \ref{lst:arrayreference}.

\begin{spacing}{1}
\begin{lstlisting}[float=htbp, caption=Ejemplo de array con tipo de variable por referencia, label=lst:arrayreference]
public class Point2D 
{ 
	public int x;
	public int y;
}

static void Main()
{
	Point2D[] x = new Point2D[5];    // Se crea un array de 5 Point2D
	x[1] = new Point2D();			 // Se asigna el valor 2 al segundo elemento del array
		
	Console.WriteLine(x[0] == null);    // true
	Console.WriteLine(x[1] == null);    // false
	Console.WriteLine(x[2] == null);    // true
	Console.WriteLine(x[3] == null);    // true
	Console.WriteLine(x[4] == null);    // true
}
\end{lstlisting}
\end{spacing}

\subsection{Clases}

En C\#, una clase es un tipo de variable por referencia, para declarar una clase basta con escribir la palabra clave \texttt{class} seguido por el nombre de la clase y finalizando por los corchetes de apertura y cierre.

Una clase puede contener uno o varios campos, estos campos son variables que pertenecen a la clase, estos campos pueden ser inicializados incluso antes de que el constructor sea ejecutado, un ejemplo de clase con un campo inicializado está en el Listado \ref{lst:fieldclass}.

\begin{spacing}{1}
\begin{lstlisting}[float=htbp, caption=Ejemplo de clase con un campo inicializado, label=lst:fieldclass]
public class Point2D 
{ 
	public int x = 10;  // El valor inicial es 10
	public int y;       // El valor inicial es el de por defecto, 0
}
\end{lstlisting}
\end{spacing}

Una clase también puede contener métodos, un método es una función que recibe cero o más parámetros y que puede devolver un parámetro. Un método se define de la siguiente manera:

\begin{spacing}{1}
\begin{lstlisting}[float=htbp, caption=Estructura de un método, label=lst:methoddeclaration]
tipo-de-retorno nombre-del-metodo(lista-de-parametros)
{
	lista-de-sentencias
}
\end{lstlisting}
\end{spacing}

La definición de un método en concreto se llama \emph{firma}, pueden cooexistir dos métodos con el mismo nombre mientras no tengan la misma firma, a esto se le llama \emph{sobrecarga} de métodos.

Dentro de cada clase hay un método especial, el constructor, este método es llamado justo después de crear un objeto de una clase con el operator \texttt{new} y una vez los campos de la clase se han inicializado, el constructor no tiene tipo de retorno y acepta tantos parámetros como sea necesario.

Un ejemplo de clase con campos, métodos sobrecargados y un constructor está en el Listado \ref{lst:classwhole}

\begin{spacing}{1}
\begin{lstlisting}[float=htbp, caption={Ejemplo de clase con campos, sobrecarga de métodos y constructor}, label=lst:classwhole]
public class Point2D 
{ 
	public int x = 10;  	// El valor inicial es 10
	public int y;       	// El valor inicial es el de por defecto, 0
	
	public Point2D(int x, int y)    // Constructor de la clase
	{
		this.x = x; 			    // Asignar el parametro x al campo x de la clase
		this.y = y; 			    // Asignar el parametro y al campo y de la clase
	}
	
	// Metodo que devuelve un Point2D y que acepta un Point2D	
	public Point2D calcularVector(Point2D p)
	{
		int vx = p.x - this.x;
		int vy = p.y - this.y;
		
		return new Point2D(vx, vy); // Devolver un nuevo Point2D
	}
	
	// Metodo sobrecargado que devuelve un Point2D y que acepta dos enteros	
	public Point2D calcularVector(int px, int py)
	{
		int vx = px - this.x;
		int vy = py - this.y;
		
		return new Point2D(vx, vy); // Devolver un nuevo Point2D
	}
}
\end{lstlisting}
\end{spacing}

\subsubsection{Herencia}

Una clase puede heredar de otra clase para ampliar o personalizar la clase base. La ventaja de la herencia es la capacidad de reutilizar la funcionalidad de la clase base en lugar de construirla desde cero. Una clase sólo puede heredar de una clase, pero puede ser heredada por muchas clases, lo que forma una jerarquía de clases.

Las referencias a una clase de una jerarquía es polimórfica. Esto significa que una variable de tipo X puede referirse a un objeto que heredó de X, como se puede ver en el Listado \ref{lst:hierarchy}, la variable \texttt{Animal} en el método \texttt{main} es polimórfica porque puede almacenar tanto una referencia a la clase \texttt{Gato} como a la clase \texttt{Pajaro}.

\begin{spacing}{1}
\begin{lstlisting}[float=htbp, caption={Ejemplo de clase con campos, sobrecarga de métodos y constructor}, label=lst:hierarchy]
public class Animal
{ 
	public int numeroDePatas;
	public bool vegetariano;
	
	public Animal(int nPatas, bool veget)
	{
		this.numeroDePatas = nPatas;
		this.vegetariano = veget;
	}	
}

public class Gato : Animal
{
	public string color;
	
	public Gato(int nPatas, bool veget, string color) :
		base(nPatas, veget) // Llamar al constructor de la clase base
	{		
		this.color = color;
	}	
}

public class Pajaro : Animal
{
	public string color;
	public int numeroDeAlas;
	
	public Pajaro(int nPatas, bool veget, string color, int nAlas) :
		base(nPatas, veget) // Llamar al constructor de la clase base
	{		
		this.color = color;
		this.numeroDeAlas = nAlas;
	}	
}

static void Main()
{
	Animal gato = new Gato(4, false, "blanco");        // Ejemplo de polimorfismo
	Animal pajaro = new Pajaro(2, false, "negro", 2);  // Ejemplo de polimorfismo
		
	Console.WriteLine("El gato tiene " + gato.numeroDePatas + " patas");
	Console.WriteLine("El pajaro tiene " + ((Pajaro) pajaro).numeroDeAlas + " alas");
	Console.WriteLine("El gato es de color " + ((Gato) gato).color);
	Console.WriteLine("El pajaro es de color " + ((Pajaro) pajaro).color);
}
\end{lstlisting}
\end{spacing}

Respecto al uso de la herencia en Unity, la clase base en Unity es \texttt{MonoBehaviour}, es decir, todas las clases base deben heredar de ella. Como nota, en Unity se favorece la composición frente a la herencia, en parte por el sistema de componentes de Unity. Lo que significa que la herencia no es tan usada como en otros entornos de desarrollo de videojuegos.

\subsection{Modificadores}

Los modificadores son palabras clave que acompañan a variables, clases y parámetros que modifican la forma de acceder a ellos u otras características como la forma en la que los métodos se comportan una vez son heredados, hay tres tipos principales de modificadores:

\paragraph{Modificadores de acceso}, que son los que controlan cómo y quién puede acceder a las variables, estos modificadores son:

\begin{itemize}
	\item \texttt{public:} No hay restricciones al acceso.
	\item \texttt{protected:} El acceso está limitado a la clase contenedora y a los tipos derivados de la clase contenedora.
	\item \texttt{internal:} El acceso está limitado al ensamblado actual, este es el modificador por defecto si no se especifica ninguno.
	\item \texttt{protected internal:} No hay restricciones al acceso.
	\item \texttt{private:} No hay restricciones al acceso.
\end{itemize}

El concepto de \emph{ensamblado} se asemeja al concepto de unidades de compilación de C, sin embargo, en C\#, los ensamblados también pueden contener recursos como iconos, imágenes, texto traducido...

\paragraph{Modificadores de clases}, que son los que controlan las propiedades de una clase, estos modificadores son:
\begin{itemize}
	\item \texttt{abstract:} Indica que la clase no tiene una implementación completa y por lo tanto no se podrán crear instancias de esa clase. La implementación debe de ser completada por clases derivadas no abstractas.
	\item \texttt{sealed:} Impide que la clase sea heredada por otras clases.
	\item \texttt{partial:} La definición de la clase está dividida en dos o más archivos, es útil para distinguir la parte de inicialización de la UI con la de lógica.
\end{itemize}

\paragraph{Modificadores de métodos}, que son los que controlan cómo se heredan los métodos de una clase a otra, estos modificadores son:

\begin{itemize}
	\item \texttt{abstract:} Indica que el método no tiene implementación y por lo tanto no se podrán crear instancias de esa clase. La implementación debe de ser completada por clases derivadas no abstractas.
	\item \texttt{sealed:} Impide que el método sea sobrescrito por otras clases derivadas.
	\item \texttt{virtual:} Indica que el método modificado podrá ser sobrescrito en una clase derivada, ya que por defecto, esto no está permitido.
	\item \texttt{override:} Permite modificar la definición de un método en una clase derivada, siempre que esto esté permitido.
\end{itemize}



\subsection{Delegados -- Programación basada en eventos}

C\# también tiene facilidades para una programación basada en eventos, este elemento se llama \texttt{delegado}.

Un delegado es un objeto que contiene una referencia a un método al que llamará cuando se le indique.

Al definir un delegado también se definirá que tipo de métodos podrá llamar, un delegado se define de la siguiente manera:

\begin{spacing}{1}
\begin{lstlisting}[float=htbp, caption=Estructura de un delegado]
delegate tipo-de-retorno nombre-del-delegado(lista-de-parametros);
\end{lstlisting}
\end{spacing}

Para añadir un método a un delegado bastará con usar el operador de asignación (\texttt{=}).

Los delegados de C\# también son multicast, es decir que un delegado no sólo almacena una única referencia a un método, si no que puede referenciar a múltiples métodos.

Un ejemplo del uso de delegados está en el Listado \ref{lst:delegate}

\begin{spacing}{1}
\begin{lstlisting}[float=htbp, caption={Ejemplo de uso de delegados unicast}, label=lst:delegate]
delegate void Saludos(string nombre);  // Declarar un delegado que no devuelve nada y que acepta un parametro string

static void saludarEspanol(string nombre)
{
	Console.WriteLine("Hola " + nombre + "!");
}

static void Main()
{
	Saludos sal = saludarEspanol;  // Crear un delegado y asignarle el metodo saludarEspanol
	sal("Javier");                 // Llamar a los metodos asignados al delegado
}
\end{lstlisting}
\end{spacing}

\subsubsection{Eventos en Unity}



En algún momento he dicho que los scripts son componentes

En algún momento deberías decir que las variables públicas se pueden ver y modificar desde el editor.

En algún momento deberías decir que es posible que los componentes tengan componentes hijos.

En algún momento deberías hablar de los eventos más importantes de los comportamientos.



\section{Uso de C\# en Unity}

La primera versión del motor de juegos Unity fue lanzada en el 2005, desde entonces mucho ha cambiado en el motor, pasando por el motor de scripting, nuevas tecnologías gráficas y el modelo de licencias.

El lenguaje de scripting de Unity pasó por varias etapas, la primera de ellas fue \emph{Boo}, un lenguaje orientado a objetos usando el \emph{Common Language Runtime} (CLR) de .NET Framework, tecnología similar a la usada por \emph{Java} en su máquina virtual. La siguiente etapa pasó por usar una variante de \emph{Javascript} como lenguaje de scripting y, por último, se pasó a usar C\# como lenguaje de scripting con \emph{Visual Studio} como uno de los IDEs soportados.

C\# es un lenguaje de programación orientado a objetos y ejecutado sobre el \emph{Common Language Runtime}, usando principalmente con tipos estáticos aunque con soporte para tipos dinámicos, también se puede enfocar a una programación basada en eventos gracias a sus \emph{delegados}.

Ha conseguido obtener la condición de estándar ISO, su última revisión fue en 2018, con referencia ISO 23270:2018\footnote{\url{https://www.iso.org/standard/75178.html}}

Actualmente Unity soporta más de 20 plataformas\footnote{\url{https://unity3d.com/es/unity/features/multiplatform}}, entre las que más destacan podemos encontrar: Windows, Linux, Mac, Playstation 4, Xbox One, Nintendo 3Ds, Oculus Rift, Android, iOS, WebGL, ARKit y ARCore.

C\# en Unity también permite interaccionar con los diferentes componentes del motor, como por ejemplo \emph{Mecanim}, su sistema de animación, su sistema de físicas o su sistema de componentes, lo que proporciona un control total sobre el motor y no sólo tareas de scripting in-game.

\subsection{Estructura de clases de Unity}

\subsection{Manejo de objetos de Unity desde C\#}

En esta sección se va a tratar c\'omo manipular objetos del mundo virtual desde un \textit{script} en \textit{Unity}. Como ya se ha dicho, los \textit{scripts} son componentes de los objetos. Desde cualquier \textit{script} podemos manipular cualquier otro objeto del mundo, o el mismo objeto. Esto último es especialmente f\'acil si queremos aplicar una transformación, pues tenemos acceso al componente \textit{Transform} como variable de clase. En \ref{lst:local} podemos ver como se puede acceder al componente \textit{transform} del objeto e invocar sus funciones para aplicarle una traslación. La variable \textit{transform} de ese listado no es necesario que se defina en ningún sitio del \textit{script}, sino que se tiene acceso a ella directamente. Esto es porque el componente \textit{Transform} es inherente a todos los objetos que están en la escena, puesto que todos los objetos deben de tener una transformación para saber dónde hay que dibujarlos. El resto de componentes (que deben ser añadidos manualmente), pueden ser accedidos simplemente con la función \textit{GetComponent<Tipo>()}. De esta forma podemos fácilmente aplicar transformaciones o fuerzas (en el caso de rigidBody) sobre el mismo objeto.

\begin{lstlisting}[float=htbp, caption={Acceso al componente transform para modificar el propio objeto}, label=lst:local]
Vector3 direction = new Vector3(mainCamera.transform.forward.x, 0, mainCamera.transform.forward.z).normalized * speed * Time.deltaTime;
Quaternion rotation = Quaternion.Euler(new Vector3(0, -transform.rotation.eulerAngles.y, 0));
transform.Translate(rotation * direction);
\end{lstlisting}

En el caso de querer modificar otro objeto, por ejemplo, en el caso de querer crear un enemigo a cierta distancia de otro, simplemente tenemos que encontrar la referencia al objeto. Esto se puede hacer muy fácilmente haciendo una variable pública y asignándola desde el editor. Sin embargo, si queremos acceder al objeto de forma dinámica o únicamente usando \textit{C\#}, se puede hacer de varias formas, todas usando funciones estáticas de la clase \textit{GameObject}:

\begin{enumerate}
\item Por nombre: usando la función \textit{GameObject.Find()}. Se puede buscar de forma directa un objeto en la escena virtual mediante su nombre, simplemente es necesario conocerlo de antemano. Este último matiz puede que sea la mayor inconveniencia, pues es probable que no sepamos de antemano el nombre del objeto, sobre todo si se ha generado dinámicamente. Esta función tratará los caracteres \"\/\" no como parte del nombre, sino como parte de una jerarquía de objetos. La documentación oficial de \textit{Unity} \footnote{\url{https://docs.unity3d.com/ScriptReference/GameObject.html}} desaconseja usar esta función cada frame.
\item Por tipo: usando la función \textit{GameObject.FindObjectOfType(Type type)}. En este caso, buscamos por tipo de objeto, teniendo un ejemplo en el listado \ref{lst:tipo}.
 Esta función retorna el \textbf{primer} objeto cargado de el tipo especificado, o \textit{null} si no existe ninguno de dicho tipo. Para conseguir un iterador de todos los objetos de ese tipo, también existe la función \textit{GameObject.FindObjectsOfType(Type type)}. La documentación oficial desaconseja utilizar estas funciones por ser más lentas que el resto.
\item Por etiqueta o \textit{tag}: usando la función \textit{GameObject.FindGameObjectWithTag(String tag)}. Esta función, al igual que la anterior, devuelve el \textbf{primer} objeto cargado con dicha tag, o \textit{null} si no encuentra ninguno. Además, lanzará una excepción si la tag no existe. Esta función es muy fácil de usar, pues \textit{Unity} tiene un sistema nativo de tags para los objetos muy simple de usar, a parte de tener tags predefinidas. El hecho de que sean simples cadenas de caracteres también lo hace más sencillo de utilizar. 
\end{enumerate}

\begin{lstlisting}[float=htbp, caption=Búsqueda de objetos por tipo, label=lst:tipo]
using UnityEngine;
using System.Collections;

// Search for any object of Type GUITexture,
// if found print its name, else print a message
// that says that it was not found.
public class ExampleClass : MonoBehaviour
{
    void Start()
    {
        GUITexture texture = (GUITexture)FindObjectOfType(typeof(GUITexture));
        if (texture)
            Debug.Log("GUITexture object found: " + texture.name);
        else
            Debug.Log("No GUITexture object could be found");
    }
}
\end{lstlisting}



\section{Ejemplos}

\section{Conclusion}



\section{Referencias}

\clearpage

%\phantomsection  % OJO: Ojo necesario con hyperref.
\addcontentsline{toc}{chapter}{\refname} % Añade la bibliografía al Índice de contenidos.
%---
% Opción 1: Bibliografía con todas las fuentes en un apartado.
%---

\nocite{*}
\printbibliography

\end{document}

% Local Variables:
% coding: utf-8
% mode: flyspell
% ispell-local-dictionary: "castellano8"
% mode: latex
% End:
